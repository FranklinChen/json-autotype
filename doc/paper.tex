\documentclass[11pt]{article}
%Gummi|065|=)
\title{\textbf{Automatic typing JSON Autotype -- way to quickly bridge typeful haven of Haskell with untyped Web APIs}}
\author{Micha\l{} J. Gajda}
\date{April 3rd, 2015}
\begin{document}

\maketitle

\abstract{
I want to tell a story about practical web-service development using type theory.

JSON is a subset of JavaScript, and convenient data format for Web 2.0 applications. Structured Web APIs often use JSON messages to transmit large answer sets that may be leveraged when building both mash-ups, and complex web services.

Since Haskell excels in precise description of such complex systems, it enjoys relatively large set of parsing, validation, and web libraries.

JSON-Autotype facilitates automatic creation of Web API interfaces in Haskell, and uses union typing for ad-hoc type discovery.

I plan to introduce problem by examples, and discuss most frequent problems of type-based interface ,,discovery''.}


\section{Introduction}

\section{Motivating examples}

\section{Schema of implementation}
dsd$\rightarrow{}$dsd
\section{Future plans}
Author plans to follow users' request, most frequent of which is to support code generation in the other languages (Scala, Java, OCaml).

It would be also interesting to compare the parsed structure with any JSON schema, that may be only partially consistent with it.

Useful extension to the type inference would be to treat tagged sum types, where a specific field of record is indicating which of the constructors are selected. Each of these constructors would then enjoy separate type.

\section{Acknowledgments}
I'd like to thank many JSON API authors that were so kind as to provide response and request examples on the web, even before thinking about JSON schema. They provided the motivation for this experiment.

\end{document}
